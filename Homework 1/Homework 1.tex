\documentclass[letterpaper, 11pt]{article}

\usepackage{tikz, listings, comment}
\usepackage[fleqn]{amsmath}
\usepackage[margin=1in]{geometry}
\usepackage{fancyhdr}

% Disable page numbers
\pagestyle{empty}

%Basic Variables (Modify as required per homework)
\def\class{CompEn 462}
\def\homeworkNumber{1}
\def\date{01.28.2025}
\def\professor{Mark Mahon}

%Code to generate new page for problem
\newcounter{problemId}
\stepcounter{problemId}
\def\newproblem{\clearpage\newpage{Problem~\arabic{problemId}\stepcounter{problemId}}\hfill\par}

%Custom Section Header Command
\newcommand{\secHeader}[1]{\vspace{2mm} \noindent \textbf{#1:}\vspace{-4mm}}

\begin{document}
%Title page
\hfill
\newline
Name: Justin Ngo
\\PSU ID: jvn5439
\\Professor: \professor
\\Class: \class
\\Date: \date

%---------------------PROBLEM 1----------------------------
\newproblem
\begin{enumerate}
    \item True or False
    \begin{enumerate}
        \item False: Assuming that $N(A) = 0$ means that there is no set of vectors that span the Null Space, then the answer is false
        \item True: the inverse of an orthogonal matrix is its transpose by definition
        \item True: Convolution operation in time domain means you multiply in the frequency domain, so $s_1(t) * s_2(t) = f_1 * f_2 * f_3 * f_4$
        \item False: A 3x3 matrix where the columns (b) form the space doesn't necessarily imply that all the values in each column is 0, which means that $a_1, a_2, a_3$ cant all be 0
        \item False: if you set v = 4, then $\left\lVert v \right\rVert = \sqrt{1+4+9+16} = \sqrt{30}$, $4/30 = 2/15 \neq 0$
    \end{enumerate}
\end{enumerate}


%---------------------PROBLEM 2----------------------------
\newproblem
\begin{gather}
    3x + 2y = 10 \notag\\
    6x + 4y = b \notag
\end{gather}

\secHeader{Infinite Solution}
\begin{align}
    3x + 2y &= 10 \notag\\
    6x + 4y &= 20 \notag\\
    0 &= 0 \notag\\
    b &= 20 \notag
\end{align}

\secHeader{No Solution}
\begin{align}
    3x + 2y &= 10 \notag\\
    6x + 4y &= 30 \notag\\
    0 &= 20 \notag\\
    b &= 30 \notag
\end{align}
so long as $b \neq 20$ there should be no solution.


%---------------------PROBLEM 3----------------------------
\newproblem
\begin{gather}
    x - y = 2 \notag\\
    x + y = 4 \notag\\
    2x + y = 8 \notag
\end{gather}
\[
\begin{bmatrix}
1 & -1 & 2 \\
1 & 1 & 4 \\
2 & 1 & 8
\end{bmatrix} 
R_1 \leftrightarrow R_3
\begin{bmatrix}
2 & 1 & 8 \\
1 & 1 & 4 \\
1 & -1 & 2
\end{bmatrix}
R_1 - R_2 \rightarrow R_1
\begin{bmatrix}
1 & 0 & 4 \\
1 & 1 & 4 \\
1 & -1 & 2
\end{bmatrix}
R_2 - R_1 \rightarrow R_2 
\]

\[
\begin{bmatrix}
    1 & 0 & 4 \\
    0 & 1 & 0 \\
    1 & -1 & 2
\end{bmatrix}
R_3 - R_1 \rightarrow R_3
\begin{bmatrix}
    1 & 0 & 4 \\
    0 & 1 & 0 \\
    0 & -1 & -2
\end{bmatrix}
R_3 + R_2 \rightarrow R_3
\begin{bmatrix}
    1 & 0 & 4 \\
    0 & 1 & 0 \\
    0 & 0 & -2
\end{bmatrix}
\Rightarrow NULL
\]

\[
A^T = 
\begin{bmatrix}
    1 & 1 & 2 \\
    -1 & 1 & 1 
\end{bmatrix}
A^TA = 
\begin{bmatrix}
    6 & 2 \\
    2 & 3
\end{bmatrix}
A^Tb = 
\begin{bmatrix}
    22 \\
    10
\end{bmatrix}
\]

\[
A^Tb = 
\begin{bmatrix}
    6 & 2 & 22\\
    2 & 3 & 10
\end{bmatrix}
R_1/6 \rightarrow R_1
\begin{bmatrix}
    1 & 1/3 & 11/3\\
    2 & 3 & 10
\end{bmatrix}
R_2 - 2R_1 \rightarrow R_2
\begin{bmatrix}
    1 & 1/3 & 1/3\\
    0 & 7/3 & 8/3
\end{bmatrix}
\]
\[
3R_2/7 \rightarrow R_2
\begin{bmatrix}
    1 & 1/3 & 11/3\\
    0 & 1 & 8/7
\end{bmatrix}
R_1 - R_2/3 \rightarrow R_1
\begin{bmatrix}
    1 & 0 & 23/7\\
    0 & 1 & 8/7
\end{bmatrix}
\]
\\
\\
\indent $x = 23/7 \\$
\indent $y = 8/7 \\$
\end{document}
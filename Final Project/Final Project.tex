\documentclass[letterpaper, 11pt]{article}

\usepackage{tikz, listings, comment}
\usepackage[fleqn]{amsmath}
\usepackage[margin=1in]{geometry}
\usepackage{fancyhdr, hyperref, pdfpages}
\usepackage{xcolor}

\definecolor{codegreen}{rgb}{0,0.6,0}
\definecolor{codegray}{rgb}{0.5,0.5,0.5}
\definecolor{codepurple}{rgb}{0.58,0,0.82}
\definecolor{backcolour}{rgb}{0.95,0.95,0.92}

% Disable page numbers
\pagestyle{empty}

%Basic Variables (Modify as required per homework)
\def\class{CompEn 462}
\def\homeworkNumber{Final Project}
\def\date{05.07.2025}
\def\professor{Mark Mahon}

% Define a custom style for listings
\lstdefinestyle{mystyle}{
    backgroundcolor=\color{backcolour},   % Set background color for the code block
    commentstyle=\color{codegreen},       % Set color for comments
    keywordstyle=\color{magenta},         % Set color for keywords
    numberstyle=\tiny\color{codegray},    % Set style for line numbers
    stringstyle=\color{codepurple},       % Set color for strings
    basicstyle=\ttfamily\footnotesize,    % Set basic style for the code
    breakatwhitespace=false,              % Do not break lines at whitespace
    breaklines=true,                      % Allow breaking of lines
    captionpos=b,                         % Set caption position to bottom
    keepspaces=true,                      % Keep spaces in text
    %numbers=left,                         % Display line numbers on the left
    numbersep=5pt,                        % Set distance between line numbers and code
    showspaces=false,                     % Do not show spaces
    showstringspaces=false,               % Do not show spaces in strings
    showtabs=false,                       % Do not show tabs
    tabsize=2                             % Set tab size to 2 spaces
}

\lstset{style=mystyle}

%Code to generate new page for problem
\newcounter{problemId}
\stepcounter{problemId}
\def\newproblem{\clearpage\newpage\noindent{Problem~\arabic{problemId}\stepcounter{problemId}}\hfill\par}

%Custom Section Header Command
\newcommand{\secHeader}[1]{\vspace{2mm} \noindent \textbf{#1:}\vspace{-4mm}}

\begin{document}
%Title page
\hfill
\newline
Name: Justin Ngo
\\PSU ID: jvn5439
\\Professor: \professor
\\Class: \class
\\Date: \date
\\Project : \homeworkNumber

%---------------------ABSTRACT---------------------
\newpage
\secHeader{Abstract}
\vspace{5mm}

This report presents the implementation and analysis of an Orthogonal Frequency Division Multiplexing (OFDM) system and 
covers several key components:

\begin{enumerate}
    \item \textbf{Modulation Schemes}: Different modulation schemes such as BPSK, $\pi$/2-BPSK, QPSK, and 64-QAM are implemented in the 
    \texttt{Modulators.py} file. These schemes are used to convert the input bit stream based off the 8-bit ASCII conversion of the string "WirelessCommunicationSystemsandSecurityJustinNgo"
    into suitable symbols ready for transmission.

    \item \textbf{OFDM Processing}: The \texttt{OFDM.py} file houses all the relevant functions used to produce an OFDM output, and handles the bulk of the processing.
    The OFDM function is responsible for running the serial-to-parallel conversion, Inverse Fast Fourier Transform (IFFT), cyclic prefix insertion, and also graphs the output.

    \item \textbf{Main Execution}: The \texttt{main.py} file generates the bit stream used for processing from the aforementioned phrase and is where the OFDM function is called 
    to run the system for each modulation scheme.

    \item \textbf{Visualization}: A sample of 2 symbols generated from the OFDM system is plotted for visualization and highlights the cyclic prefix and symbol boundaries. 
\end{enumerate}






\end{document}